\documentclass[a4]{article}
\pagestyle{myheadings}

%%%%%%%%%%%%%%%%%%%
% Packages/Macros %
%%%%%%%%%%%%%%%%%%%
\usepackage{mathrsfs}


\usepackage{fancyhdr}
\pagestyle{fancy}
\lhead{}
\chead{}
\rhead{}
\lfoot{}
\cfoot{} 
\rfoot{\normalsize\thepage}
\renewcommand{\headrulewidth}{0pt}
\renewcommand{\footrulewidth}{0pt}
\newcommand{\RomanNumeralCaps}[1]
{\MakeUppercase{\romannumeral #1}}

\usepackage{amssymb,latexsym}  % Standard packages
\usepackage[utf8]{inputenc}
\usepackage[russian]{babel}
\usepackage{MnSymbol}
\usepackage{amsmath,amsthm}
\usepackage{indentfirst}
\usepackage{graphicx}%,vmargin}
\usepackage{graphicx}
\graphicspath{{pictures/}} 
\usepackage{verbatim}
\usepackage{color}









\DeclareGraphicsExtensions{.pdf,.png,.jpg}% -- настройка картинок

\usepackage{epigraph} %%% to make inspirational quotes.
\usepackage[all]{xy} %for XyPic'a
\usepackage{color} 
\usepackage{amscd} %для коммутативных диграмм


\newtheorem{Lemma}{Лемма}[section]
\newtheorem{Proposition}{Предложение}[section]
\newtheorem{Theorem}{Теорема}[section]
\newtheorem{Corollary}{Следствие}[section]
\newtheorem{Remark}{Замечание}[section]
\newtheorem{Definition}{Определение}[section]
\newtheorem{Designations}{Обозначение}[section]




%%%%%%%%%%%%%%%%%%%%%%%% 
%Сношение с оглавлением% 
%%%%%%%%%%%%%%%%%%%%%%%% 
\usepackage{tocloft} 
\renewcommand{\cftdotsep}{2} %частота точек
\renewcommand\cftsecleader{\cftdotfill{\cftdotsep}}
\renewcommand{\cfttoctitlefont}{\hspace{0.38\textwidth} \LARGE\bfseries} 
\renewcommand{\cftsecaftersnum}{.}
\renewcommand{\cftsubsecaftersnum}{.}
\renewcommand{\cftbeforetoctitleskip}{-1em} 
\renewcommand{\cftaftertoctitle}{\mbox{}\hfill \\ \mbox{}\hfill{\footnotesize Стр.}\vspace{-0.5em}} 
\renewcommand{\cftsubsecfont}{\hspace{1pt}} 
\renewcommand{\cftparskip}{3mm} %определяет величину отступа в оглавлении
\setcounter{tocdepth}{5} 




\addtolength{\textwidth}{0.7in}
\textheight=630pt
\addtolength{\evensidemargin}{-0.4in}
\addtolength{\oddsidemargin}{-0.4in}
\addtolength{\topmargin}{-0.4in}

\newcommand{\empline}{\mbox{}\newline} 
\newcommand{\likechapterheading}[1]{ 
	\begin{center} 
		\textbf{\MakeUppercase{#1}} 
	\end{center} 
	\empline} 

\makeatletter 
\renewcommand{\@dotsep}{2} 
\newcommand{\l@likechapter}[2]{{\bfseries\@dottedtocline{0}{0pt}{0pt}{#1}{#2}}} 
\makeatother 
\newcommand{\likechapter}[1]{ 
	\likechapterheading{#1} 
	\addcontentsline{toc}{likechapter}{\MakeUppercase{#1}}} 





\usepackage{xcolor}
\usepackage{hyperref}
\definecolor{linkcolor}{HTML}{000000} % цвет ссылок
\definecolor{urlcolor}{HTML}{AA1622} % цвет гиперссылок

\hypersetup{pdfstartview=FitH,  linkcolor=linkcolor,urlcolor=urlcolor, colorlinks=true}



\def \newstr {\medskip \par \noindent} 



\begin{document}
	\def\contentsname{\LARGE{Содержание}}
	\thispagestyle{empty}
	\begin{center} 
		\vspace{2cm} 
		{\Large \sc Санкт-Петербургский Политехнический Университет}\\
		\vspace{2mm}
		{\Large\sc Петра Великого}\\
		\vspace{1cm}
		{\large \sc Институт прикладной математики и механики\\ 
			\vspace{0.5mm}
			\textsc{}}\\ 
		\vspace{0.5mm}
		{\large\sc Кафедра $"$Прикладная математика$"$}\\
		\vspace{15mm}
		
		
		{\sc \textbf{Отчёт\\
			Курсовая работа\\
			по дисциплине\\
			"Математическая статистика"}
			\vspace{6mm}
			
		}
		\vspace*{2mm}
		
		
		\begin{flushleft}
			\vspace{4cm}
			\sc Выполнил студент:\\
			\sc Салихов С.Р.\\
			\sc группа: 3630102/70401\\
			\vspace{1cm}
			\sc Проверил:\\
			\sc к.ф-м.н., доцент\\
			\sc Баженов Александр Николавич
			\vspace{20mm}
		\end{flushleft}
	\end{center} 
	\begin{center}
		\vfill {\large\textsc{Санкт-Петербург}}\\ 
		2020 г.
	\end{center}
	
	\newpage
	\pagestyle{plain}
	
	%\begin{center}
	%\begin{abstract} 
	
	%\end{abstract}
	
	%\end{center}
	
	\newpage
	\tableofcontents{}
	\newpage
	\listoffigures
	\newpage
	
	
	\section{Постановка задачи}
	Даны показания 4-х датчиков, регистрирующих мягкое рентгеновское излучение плазмы в пяти экспериментах. В показаниях датчиков иногда наблюдаются пилообразные колебания, предшествующие срыву плазмы. Важно уметь вовремя детектировать такие колебания, чтобы предотвращать срыв плазмы. В связи с этим требуется:\\
	
	1)Представить алгоритм выделения пилобразных колебаний.\\
	
	2)Оценить частоту пилобразных колебаний.\\
	
	3)Построить гистограммы частот для различных датчиков.\\
	
	4)Выяснить наличие корреляции у различных датчиков на разных временных этапах пилобразных колебаний.\\
	
	\section{Подготовка данных}
	Данные представлены в бинарном сжатом виде. Декодирование данных производится с помощью Python и библиотеки pyGlobus. Далее из декодировнных данных извлекаются временные последовательности показаний датчиков, которые сохраняются в массивах numpy.\\
	
	Представлены наборы данных для 5-и экспериментов: 38993, 38994, 38995, 38996, 38998.\\
	
	Каждый набор содержит измерения 4-х датчиков: SXR 15 мкм, SXR 27 мкм, SXR 50 мкм, SXR 80 мкм.
	
	\section{Алгоритм выделения пилобразных колебаний}
	\subsection{Шаги алгоритма}
		0)Выбирается сигнал для анализа.\\
		
		1)Находим участок сигнала, который не является квазистационарным. Данный участок выделяется сравнением значения отсчёта с его средним значением по всему сигналу(данный шаг алгоритма не применяется в онлайн режиме).\\
		
		2)Для удаления низкачастотных составляющих применяем фильтр высоких частот.\\
		
		3)Находим 1-ю про-ю спрямленного сигнала путём применения сглаживающего цифрового дифференцирующего фильтра(ЦДФ):\\
		$y(n) = \sum_{k=1}^{M} \frac{1}{M(M+1)} [x(n+k) - x(n - k)]$, $M$ = 2 - порядок фильтра, $x(n)$ - значение n-ого отсчёта входного сигнала, а $y(n)$ - значение n-го отсчёта входного сигнала.\\
		
		4)Берём модуль от первой производной спрямленного сигнала. И применяем фильтр низких частот, для удаления высокочастотного шума.\\
		
		5)По установленному порогу определяем пилобразные колебания.\\
	
	\newpage
	\subsection{Иллюстрация шагов алгоритма на примере с данными 38998, для датчика SXR 27 мкм}
		\begin{center}
			\newpage
			\begin{figure}[h!]
				\includegraphics[width=\textwidth]{output/first/pic0.png}\caption[Исходный сигнал(0-й шаг алгоритма)]{Исходный сигнал(0-й шаг алгоритма)}
			\end{figure}
			\newpage
			\begin{figure}[h!]
				\includegraphics[width=\textwidth]{output/first/pic1.png}\caption[Результаты после 1-го шага алгоритма]{Результаты после 1-го шага алгоритма}
			\end{figure}
		\newpage
			\begin{figure}[h!]
				\includegraphics[width=\textwidth]{output/first/pic2.png}\caption[Результат после 2-го шага алгоритма, частота среза = 850, для фильтра высоких частот]{Результат после 2-го шага алгоритма, частота среза = 850, для фильтра высоких частот}
			\end{figure}
			\newpage
			\begin{figure}[h!]
				\includegraphics[width=\textwidth]{output/first/pic3.png}\caption[Результат после 3-го шага алгоритма, порядок фильтра = 50]{Результат после 3-го шага алгоритма, порядок фильтра = 50}
			\end{figure}
			\newpage
			\begin{figure}[h!]
				\includegraphics[width=\textwidth]{output/first/pic4.png}\caption[Результат после 4-го шага алгоритма, частота среза = 4250, для фильтра низких частот]{Результат после 4-го шага алгоритма, частота среза = 4250, для фильтра низких частот}
			\end{figure}
		\newpage
			\begin{figure}[h!]
				\includegraphics[width=\textwidth]{output/first/pic5.png}\caption[Результат после 5-го шага алгоритма, 0.00 - 0.26-с - границы пилобразного участка, 0.00004 - установленный порог]{Результат после 5-го шага алгоритма, 0.00 - 0.26-с - границы пилобразного участка, 0.00004 - установленный порог}
			\end{figure}
				
		\end{center}
		\newpage
		\subsection{Промежуточные выводы по алгоритму}
			1)Время выполнения для тестового примера $\approx$ 80 сек.(в данном времени не учитывалось время построения и сохранения графиков), основное время работы занимает 3-й шаг алгоритма. Таким образом, для отслеживания пилобразных колебаний в реальном времени нужна большая вычислительная мощность, либо необходимо представить данный алгоритм на более низком языке программирования.\\
			
			2)Для выполнения 5-го шага алгоритма необходимо на тестовых данных определить допустимое пороговое значение, чтобы в дальнейшем использовать его на будущих вычислениях.\\
		
		\newpage
		\section{Алгоритм для оценивания частоты пилобразных колебаний}
			\subsection{Шаги алгоритма}
				0)Выделяем участок с пилобразными коллебаниями.\\
				
				1)Применяем фильтр высоких частот и удаляем высокочастотные шумы с помощью фильтра низких частот.\\
				
				2)Ищем точки пересечения сигнала с осью абцисс. Для получения мгновенных периодов колебаний вычитаем полученные значения, через 1-н.\\
				
				3)Из периодов (полученных на шаге 2), получаем мгновенные частоты - функция частоты от времени.\\
				
				4)Применяем любой сглаживающий фильтр. Т.к. наша функция зависит от времени, то лучше выбрать скользящие среднее.
			\newpage
			\subsection{Иллюстрация шагов алгоритма на примере с данными 38998, для датчика SXR 27 мкм}
			\begin{center}
				\newpage
				\begin{figure}[h!]
					\includegraphics[width=\textwidth]{output/second/pic1.png}\caption[Результаты после 0-го шага алгоритма]{Результаты после 0-го шага алгоритма}
				\end{figure}
				\newpage
				\begin{figure}[h!]
					\includegraphics[width=\textwidth]{output/second/pic3.png}\caption[Результат после 1-го шага алгоритма, частота для фильтра низких частот = 2500, частота для фильтра высоких частот = 1000]{Результат после 1-го шага алгоритма, частота для фильтра низких частот = 2500, частота для фильтра высоких частот = 1000}
				\end{figure}
				\newpage
				\begin{figure}[h!]
					\includegraphics[width=\textwidth]{output/second/pic4.png} \caption[Результат после 4-го шага алгоритма]{Результат после 4-го шага алгоритма}
				\end{figure}
				
			\end{center}
		\newpage
		\subsection{Промежуточные выводы по алгоритму}
			1)Время работы алгоритма меньше 1сек(при исключении работы с диском и построения графиков).
			
	
	\section{Графики функции частоты от времени для всех датчиков экспериментов}
	\begin{center}\newpage
	\begin{figure}[h!]
		\includegraphics[width=\textwidth]{output/third/pic0.png}\caption[График функции частоты от времени для участка пилобразных колебаний для всех датчиков в эксперименте 38993]{График функции частоты от времени для участка пилобразных колебаний для всех датчиков в эксперименте 38993}
	\end{figure}\newpage
	\begin{figure}[h!]
		\includegraphics[width=\textwidth]{output/third/pic1.png}\caption[График функции частоты от времени для участка пилобразных колебаний для всех датчиков в эксперименте 38994]{График функции частоты от времени для участка пилобразных колебаний для всех датчиков в эксперименте 38994}
	\end{figure}\newpage
	\begin{figure}[h!]
		\includegraphics[width=\textwidth]{output/third/pic2.png}\caption[График функции частоты от времени для участка пилобразных колебаний для всех датчиков в эксперименте 38995]{График функции частоты от времени для участка пилобразных колебаний для всех датчиков в эксперименте 38995}
	\end{figure}\newpage
	\begin{figure}[h!]
		\includegraphics[width=\textwidth]{output/third/pic3.png}\caption[График функции частоты от времени для участка пилобразных колебаний для всех датчиков в эксперименте 38996]{График функции частоты от времени для участка пилобразных колебаний для всех датчиков в эксперименте 38996}
	\end{figure}\newpage
	\begin{figure}[h!]
		\includegraphics[width=\textwidth]{output/third/pic4.png}\caption[График функции частоты от времени для участка пилобразных колебаний для всех датчиков в эксперименте 38998]{График функции частоты от времени для участка пилобразных колебаний для всех датчиков в эксперименте 38998}
	\end{figure}
\end{center}
		\newpage
	\section{Вывод}
		1)Полученные графики свидетельствуют о том, что алгоритм детектирования пилобразных коллебаний обладает достаточной точностью для использования на практике.\\
		
		2)Ориентируясь на данные полученные из примеров видно, что алгоритм для выделения частот пилобразных коллебаний также даёт удовлетворительный результат.\\
		
		3)Также видно, что из за различного характера коллебаний необходимо больше данных для определения порогового значения для 1-го алгоритма.
		\newpage
	\section{Список литературы}
	[1]\href{https://ru.wikipedia.org/wiki/%D0%A1%D0%BA%D0%BE%D0%BB%D1%8C%D0%B7%D1%8F%D1%89%D0%B0%D1%8F_%D1%81%D1%80%D0%B5%D0%B4%D0%BD%D1%8F%D1%8F#%D0%9F%D1%80%D0%BE%D1%81%D1%82%D0%BE%D0%B5_%D1%81%D0%BA%D0%BE%D0%BB%D1%8C%D0%B7%D1%8F%D1%89%D0%B5%D0%B5_%D1%81%D1%80%D0%B5%D0%B4%D0%BD%D0%B5%D0%B5}{Скользящая средняя} \\
	
	[2]https://www.academia.edu/21643111/Development\_of\_real-time\_plasma\_analysis\_and\_control\_algorithms\_for\_the\_TCV\_tokamak\_using\_Simulink\\
	
	[3]https://numpy.org/ \\
	
	[4]https://matplotlib.org/ \\
	
	[5]https://github.com/dev0x13/globus-plasma/releases/tag/v0.1.1 \\
	
	[6]\href{https://ru.wikipedia.org/wiki/%D0%93%D0%B0%D1%80%D0%BC%D0%BE%D0%BD%D0%B8%D1%87%D0%B5%D1%81%D0%BA%D0%B8%D0%B5_%D0%BA%D0%BE%D0%BB%D0%B5%D0%B1%D0%B0%D0%BD%D0%B8%D1%8F}{Гармонические колебания} \\

	[7]\href{https://cmi.to/%D1%84%D0%B8%D0%BB%D1%8C%D1%82%D1%80%D1%8B/}{Цифровые фильтры}
\end{document}