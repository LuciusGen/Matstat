\documentclass[a4]{article}
\pagestyle{myheadings}

%%%%%%%%%%%%%%%%%%%
% Packages/Macros %
%%%%%%%%%%%%%%%%%%%
\usepackage{mathrsfs}


\usepackage{fancyhdr}
\pagestyle{fancy}
\lhead{}
\chead{}
\rhead{}
\lfoot{}
\cfoot{} 
\rfoot{\normalsize\thepage}
\renewcommand{\headrulewidth}{0pt}
\renewcommand{\footrulewidth}{0pt}
\newcommand{\RomanNumeralCaps}[1]
{\MakeUppercase{\romannumeral #1}}

\usepackage{amssymb,latexsym}  % Standard packages
\usepackage[utf8]{inputenc}
\usepackage[russian]{babel}
\usepackage{MnSymbol}
\usepackage{amsmath,amsthm}
\usepackage{indentfirst}
\usepackage{graphicx}%,vmargin}
\usepackage{graphicx}
\graphicspath{{pictures/}} 
\usepackage{verbatim}
\usepackage{color}









\DeclareGraphicsExtensions{.pdf,.png,.jpg}% -- настройка картинок

\usepackage{epigraph} %%% to make inspirational quotes.
\usepackage[all]{xy} %for XyPic'a
\usepackage{color} 
\usepackage{amscd} %для коммутативных диграмм


\newtheorem{Lemma}{Лемма}[section]
\newtheorem{Proposition}{Предложение}[section]
\newtheorem{Theorem}{Теорема}[section]
\newtheorem{Corollary}{Следствие}[section]
\newtheorem{Remark}{Замечание}[section]
\newtheorem{Definition}{Определение}[section]
\newtheorem{Designations}{Обозначение}[section]




%%%%%%%%%%%%%%%%%%%%%%%% 
%Сношение с оглавлением% 
%%%%%%%%%%%%%%%%%%%%%%%% 
\usepackage{tocloft} 
\renewcommand{\cftdotsep}{2} %частота точек
\renewcommand\cftsecleader{\cftdotfill{\cftdotsep}}
\renewcommand{\cfttoctitlefont}{\hspace{0.38\textwidth} \LARGE\bfseries} 
\renewcommand{\cftsecaftersnum}{.}
\renewcommand{\cftsubsecaftersnum}{.}
\renewcommand{\cftbeforetoctitleskip}{-1em} 
\renewcommand{\cftaftertoctitle}{\mbox{}\hfill \\ \mbox{}\hfill{\footnotesize Стр.}\vspace{-0.5em}} 
\renewcommand{\cftsubsecfont}{\hspace{1pt}} 
\renewcommand{\cftparskip}{3mm} %определяет величину отступа в оглавлении
\setcounter{tocdepth}{5} 




\addtolength{\textwidth}{0.7in}
\textheight=630pt
\addtolength{\evensidemargin}{-0.4in}
\addtolength{\oddsidemargin}{-0.4in}
\addtolength{\topmargin}{-0.4in}

\newcommand{\empline}{\mbox{}\newline} 
\newcommand{\likechapterheading}[1]{ 
	\begin{center} 
		\textbf{\MakeUppercase{#1}} 
	\end{center} 
	\empline} 

\makeatletter 
\renewcommand{\@dotsep}{2} 
\newcommand{\l@likechapter}[2]{{\bfseries\@dottedtocline{0}{0pt}{0pt}{#1}{#2}}} 
\makeatother 
\newcommand{\likechapter}[1]{ 
	\likechapterheading{#1} 
	\addcontentsline{toc}{likechapter}{\MakeUppercase{#1}}} 





\usepackage{xcolor}
\usepackage{hyperref}
\definecolor{linkcolor}{HTML}{000000} % цвет ссылок
\definecolor{urlcolor}{HTML}{AA1622} % цвет гиперссылок

\hypersetup{pdfstartview=FitH,  linkcolor=linkcolor,urlcolor=urlcolor, colorlinks=true}



\def \newstr {\medskip \par \noindent} 



\begin{document}
	\def\contentsname{\LARGE{Содержание}}
	\thispagestyle{empty}
	\begin{center} 
		\vspace{2cm} 
		{\Large \sc Санкт-Петербургский Политехнический Университет}\\
		\vspace{2mm}
		{\Large\sc Петра Великого}\\
		\vspace{1cm}
		{\large \sc Институт прикладной математики и механики\\ 
			\vspace{0.5mm}
			\textsc{}}\\ 
		\vspace{0.5mm}
		{\large\sc Кафедра $"$Прикладная математика$"$}\\
		\vspace{15mm}
		
		
		{\sc \textbf{Отчёт\\
			Лабораторная работа №$1$\\
			по дисциплине\\
			"Математическая статистика"}
			\vspace{6mm}
			
		}
		\vspace*{2mm}
		
		
		\begin{flushleft}
			\vspace{4cm}
			\sc Выполнил студент:\\
			\sc Салихов С.Р.\\
			\sc группа: 3630102/70401\\
			\vspace{1cm}
			\sc Проверил:\\
			\sc к.ф-м.н., доцент\\
			\sc Баженов Александр Николавич
			\vspace{20mm}
		\end{flushleft}
	\end{center} 
	\begin{center}
		\vfill {\large\textsc{Санкт-Петербург}}\\ 
		2020 г.
	\end{center}
	
	\newpage
	\pagestyle{plain}
	
	%\begin{center}
	%\begin{abstract} 
	
	%\end{abstract}
	
	%\end{center}
	
	\newpage
	\tableofcontents{}
	\newpage
	\listoftables
	\newpage
	
	
	\section{Постановка задачи}
	
	Для 5-ти рапределений:\\
		Нормальное распределение $N(x,0,1)$\\
		Распределение Коши $C(x,0,1)$\\
		Распределение Лапласа $L( x,0,\frac{1}{\sqrt{2}})$\\
		Распределение Пуассона $P(k, 10)$\\
		Равномерное Распределение $U(x,-\sqrt{3}, \sqrt{3})$\\
		
		Сгенерировать выборки размером 10, 50 и 1000 элементов.
		Для каждой выборки вычислить $\overline{x},\; med\; x,\; Z_R,\; Z_Q,\; Z_{tr},$ при $r = \frac{n}{4}.$.
		
	
	\section{Теория}
		\begin{enumerate}
			\item Выборочное среднее \cite{average}:
			\begin{equation}\label{eqn:average}
			\overline{x} = \frac{1}{n}\sum_{i=1}^n x_i \hfill  
			\end{equation}
			\item Выборочная медиана \cite{med}:
			\begin{equation}
			med\; x = \begin{cases}
			x_{k+1}, & n = 2k+1\\
			\frac{1}{2}\left(x_k+x_{k+1}\right), & n = 2k
			\end{cases} \hfill  \label{eqn:med}
			\end{equation}
			\item Полусумма экстремальных значений \cite{mean_extr}:
			\begin{equation}
			Z_R = \frac{1}{2}\left(x_1+x_n\right) \hfill  \label{eqn:mean_extr}
			\end{equation}
			\item Полусумма квартилей \cite{quartiles}:
			\begin{equation}
			Z_Q = \frac{1}{2}\left(Z_{\frac{1}{4}}+Z_{\frac{3}{4}}\right) \hfill  \label{eqn:quartiles}
			\end{equation}
			\item Усечённое среднее \cite{cut_mean}:
			\begin{equation}
			Z_{tr} = \frac{1}{n - 2r}\sum_{i=r+1}^{n-r} x_i \hfill  \label{eqn:cut_mean}
			\end{equation}
		\end{enumerate}	
		
	\section{Реализация}
	Для генерации выборки был использован $Python\;3.7$: модуль $random$ библиотеки $numpy$ для генерации случайных чисел с различными распределениями.\\
	После вычисления характеристик положения $1000$ раз находится среднее значение и дисперсия: 
	\begin{equation}
	E(z) = \frac{1}{n}\sum_{i=1}^n z_i
	\end{equation} 
	\begin{equation}
	D(z) = E\left(z^2\right) - E^2(z)
	\end{equation}
	
	\section{Результаты}
		\begin{table}[h]
			\caption{ Стандартное нормальное распределение.}
			\begin{center}
				\begin{tabular}{|c|c|c|c|c|c|}
					\hline
					$n = 10$ & average & med & $Z_R$ & $Z_Q$ & $Z_{tr},\;r=\frac{n}{4}$\\
					\hline
					$E =$ & $0,012$ & $-0.017$ &        $-0.012$ &       $0.007$ &         $-0.011$\\
					\hline
					$D =$ & $0.099329$&         $0.136603$        & $0.180949$     &    $0.117795$      &   $0.113781$\\
					\hline
					$n = 50$ & average & med & $Z_R$ & $Z_Q$ & $Z_{tr},\;r=\frac{n}{4}$\\
					\hline
					$E =$ & $0.001$ & $-0.003$ & $0,009$ & $-0.002$ & $0,007$\\
					\hline
					$D =$ & $0.019784$&         0.031479        & 0.119618      &   0.024300       &  0.023070\\
					\hline
					$n = 1000$ & average & med & $Z_R$ & $Z_Q$ & $Z_{tr},\;r=\frac{n}{4}$\\
					\hline
					$E =$ & -0.002       & -0.002&        0.007&         -0.001&        0.000\\
					\hline
					$D =$ & 0.001009        & 0.001542       &  0.068695       &  0.001319       &  0.001158\\
					\hline
				\end{tabular}
			\end{center}
		\end{table}
		\newpage
		\begin{table}[h]
			\caption{ Стандартное распределение Коши.}
			\begin{center}
				\begin{tabular}{|c|c|c|c|c|c|}
					\hline
					$n = 10$   & average & med & $Z_R$ & $Z_Q$ & $Z_{tr},\;r=\frac{n}{4}$\\ \hline
					$E =$      & -0.679112 &       0.015196    &     -2.113598  &      -0.002701        &-0.021571\\ \hline
					$D =$       &	412.916940  &     0.341791 &        8385.756871    &  0.803734       &  0.477004\\    \hline
					
					$n = 50$   & average & med & $Z_R$ & $Z_Q$ & $Z_{tr},\;r=\frac{n}{4}$\\ \hline
					$E =$   &-22.042147      & -0.005189       & 8.862238       &  -0.010248      &  0.003119\\   \hline
					$D =$      & 474719.963665   & 0.052230        & 1746206.475624  & 0.102886        & 0.061232  \\   \hline 
					
					$n = 1000$   & average & med & $Z_R$ & $Z_Q$ & $Z_{tr},\;r=\frac{n}{4}$\\ \hline
					$E =$      & 0.283247 &        0.000389     &    1150.281859 &     -0.001524    &    -0.002924\\  \hline
					$D =$    &  272.360308      & 0.002577        & 2196736948.38 & 0.004952 &        0.002573\\    
					\hline
				\end{tabular}
			\end{center}
		\end{table}
		\newpage
		\begin{table}[h]
			\caption{ Распределение Лапласа.}
			\begin{center}
				\begin{tabular}{|c|c|c|c|c|c|}
					\hline
					$n = 10$    & average & med & $Z_R$ & $Z_Q$ & $Z_{tr},\;r=\frac{n}{4}$\\ \hline 
					$E = $    &  	-0.001    &    -0.010 &       0.026      &   -0.015      &  -0.002   \\ \hline
					$D = $     & 0.100732      &   0.069101        & 0.425729      &   0.094138       &  0.071388    \\ \hline
					
					$n = 50$  & average & med & $Z_R$ & $Z_Q$ & $Z_{tr},\;r=\frac{n}{4}$\\ \hline
					$E = $     & -0.004       & -0.000       & 0.022 &         0.005 &         0.004   \\ \hline
					$D =$      & 	0.019090    &     0.013506  &       0.401127      &   0.020183      &   0.012568  \\ \hline
					
					$n = 1000$   & average & med & $Z_R$ & $Z_Q$ & $Z_{tr},\;r=\frac{n}{4}$\\ \hline
					$E =$      & 	0.000    &     0.000  &       -0.036     &   0.001     &    0.000     \\ \hline
					$D = $     & 0.001024        & 0.000521       &  0.406238      &   0.000957       &  0.000602 \\ 
					\hline
				\end{tabular}
			\end{center}
		\end{table}
		
		\begin{table}[h]
			\caption{ Равномерное распределение.}
			\begin{center}
				\begin{tabular}{|c|c|c|c|c|c|}
					\hline
					$n = 10$  & average & med & $Z_R$ & $Z_Q$ & $Z_{tr},\;r=\frac{n}{4}$\\ \hline
					$E =$       &	0.007  &       -0.004 &       -0.000       & -0.0212      &  0.000\\ \hline  
					$D =$       &	0.098426     &    0.220679 &        0.041760      &   0.139387       &  0.158571  \\ \hline
					
					$n = 50$  & average & med & $Z_R$ & $Z_Q$ & $Z_{tr},\;r=\frac{n}{4}$\\ \hline
					$E =$       & 0.001 &        0.005      &   0.000     &    -0.004 &       -0.009    \\ \hline
					$D =$       &	0.021157      &   0.056003  &       0.002331      &   0.029958       &  0.034239\\ \hline
					
					$n = 1000$  & average & med & $Z_R$ & $Z_Q$ & $Z_{tr},\;r=\frac{n}{4}$\\ \hline
					$E =$     &  	0.000    &     -0.001 &        0.000       &  0.002       &  0.000\\ \hline
					$D =$    &   	0.001032      &   0.002946    &     0.000006      &   0.001437       &  0.002022  \\
					\hline
				\end{tabular}
			\end{center}
		\end{table}
		
		\begin{table}[h]
			\caption{ Распределение Пуассона.}
			\begin{center}
				\begin{tabular}{|c|c|c|c|c|c|}
					\hline
					$n = 10$   & average & med & $Z_R$ & $Z_Q$ & $Z_{tr},\;r=\frac{n}{4}$\\ \hline
					$E =$     & 10.040        9.899        & 10.319     &   9.912      &   9.898 \\ \hline
					$D =$     &  	1.005791   &      1.493150  &       1.793239      &   1.111356      &   1.144207\\ \hline
					
					$n = 50$   & average & med & $Z_R$ & $Z_Q$ & $Z_{tr},\;r=\frac{n}{4}$\\ \hline
					$E =$      & 	9.969     &    9.835  &       10.748     &   9.928     &    9.874    \\ \hline
					$D =$       &	0.196334     &   0.356690   &      1.181998      &   0.281660      &   0.267151  \\ \hline
					
					$n = 1000$   & average & med & $Z_R$ & $Z_Q$ & $Z_{tr},\;r=\frac{n}{4}$\\ \hline
					$E =$      & 10.000        9.997     &    11.657    &    9.995  &       9.853\\ \hline
					$D =$      & 	0.008918   &      0.002991  &       0.643851      &   0.002528      &   0.011820  \\
					\hline
				\end{tabular}
			\end{center}
		\end{table}
		\newpage
			
	\section{Обсуждение}
		\par При вычислении средних значений пришлось отбрасывать некоторое число знаков после запятой, так как получаемая дисперсия не могла гарантировать получаемое точное значение. \\
		Иными словами дисперсия может гарантировать порядок точности среднего значения только до первого значащего знака после запятой в дисперсии включительно.\\ Единственным исключением [в отбрасывании знаков после запятой] стало стандартное распределение Коши, так как оно имеет бесконечную дисперсию, а значит не может гарантировать никакой точности.
		
		
		
	\section{Литература}
	
	\href{https://physics.susu.ru/vorontsov/language/numpy.html}{Модуль numpy}
	
	\section{Приложения}
	
	\href{https://github.com/LuciusGen/Matstat/tree/master/Lab2/lab2.py}{Код лаборатрной}
	
	\href{https://github.com/LuciusGen/Matstat/tree/master/Lab2/document.tex}{Код отчёта}
	
\end{document}