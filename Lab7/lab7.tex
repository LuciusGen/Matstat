\documentclass[a4]{article}
\pagestyle{myheadings}

%%%%%%%%%%%%%%%%%%%
% Packages/Macros %
%%%%%%%%%%%%%%%%%%%
\usepackage{mathrsfs}


\usepackage{fancyhdr}
\pagestyle{fancy}
\lhead{}
\chead{}
\rhead{}
\lfoot{}
\cfoot{} 
\rfoot{\normalsize\thepage}
\renewcommand{\headrulewidth}{0pt}
\renewcommand{\footrulewidth}{0pt}
\newcommand{\RomanNumeralCaps}[1]
{\MakeUppercase{\romannumeral #1}}

\usepackage{amssymb,latexsym}  % Standard packages
\usepackage[utf8]{inputenc}
\usepackage[russian]{babel}
\usepackage{MnSymbol}
\usepackage{amsmath,amsthm}
\usepackage{indentfirst}
\usepackage{graphicx}%,vmargin}
\usepackage{graphicx}
\graphicspath{{pictures/}} 
\usepackage{verbatim}
\usepackage{color}









\DeclareGraphicsExtensions{.pdf,.png,.jpg}% -- настройка картинок

\usepackage{epigraph} %%% to make inspirational quotes.
\usepackage[all]{xy} %for XyPic'a
\usepackage{color} 
\usepackage{amscd} %для коммутативных диграмм


\newtheorem{Lemma}{Лемма}[section]
\newtheorem{Proposition}{Предложение}[section]
\newtheorem{Theorem}{Теорема}[section]
\newtheorem{Corollary}{Следствие}[section]
\newtheorem{Remark}{Замечание}[section]
\newtheorem{Definition}{Определение}[section]
\newtheorem{Designations}{Обозначение}[section]




%%%%%%%%%%%%%%%%%%%%%%%% 
%Сношение с оглавлением% 
%%%%%%%%%%%%%%%%%%%%%%%% 
\usepackage{tocloft} 
\renewcommand{\cftdotsep}{2} %частота точек
\renewcommand\cftsecleader{\cftdotfill{\cftdotsep}}
\renewcommand{\cfttoctitlefont}{\hspace{0.38\textwidth} \LARGE\bfseries} 
\renewcommand{\cftsecaftersnum}{.}
\renewcommand{\cftsubsecaftersnum}{.}
\renewcommand{\cftbeforetoctitleskip}{-1em} 
\renewcommand{\cftaftertoctitle}{\mbox{}\hfill \\ \mbox{}\hfill{\footnotesize Стр.}\vspace{-0.5em}} 
\renewcommand{\cftsubsecfont}{\hspace{1pt}} 
\renewcommand{\cftparskip}{3mm} %определяет величину отступа в оглавлении
\setcounter{tocdepth}{5} 




\addtolength{\textwidth}{0.7in}
\textheight=630pt
\addtolength{\evensidemargin}{-0.4in}
\addtolength{\oddsidemargin}{-0.4in}
\addtolength{\topmargin}{-0.4in}

\newcommand{\empline}{\mbox{}\newline} 
\newcommand{\likechapterheading}[1]{ 
	\begin{center} 
		\textbf{\MakeUppercase{#1}} 
	\end{center} 
	\empline} 

\makeatletter 
\renewcommand{\@dotsep}{2} 
\newcommand{\l@likechapter}[2]{{\bfseries\@dottedtocline{0}{0pt}{0pt}{#1}{#2}}} 
\makeatother 
\newcommand{\likechapter}[1]{ 
	\likechapterheading{#1} 
	\addcontentsline{toc}{likechapter}{\MakeUppercase{#1}}} 





\usepackage{xcolor}
\usepackage{hyperref}
\definecolor{linkcolor}{HTML}{000000} % цвет ссылок
\definecolor{urlcolor}{HTML}{AA1622} % цвет гиперссылок

\hypersetup{pdfstartview=FitH,  linkcolor=linkcolor,urlcolor=urlcolor, colorlinks=true}



\def \newstr {\medskip \par \noindent} 



\begin{document}
	\def\contentsname{\LARGE{Содержание}}
	\thispagestyle{empty}
	\begin{center} 
		\vspace{2cm} 
		{\Large \sc Санкт-Петербургский Политехнический Университет}\\
		\vspace{2mm}
		{\Large\sc Петра Великого}\\
		\vspace{1cm}
		{\large \sc Институт прикладной математики и механики\\ 
			\vspace{0.5mm}
			\textsc{}}\\ 
		\vspace{0.5mm}
		{\large\sc Кафедра $"$Прикладная математика$"$}\\
		\vspace{15mm}
		
		
		{\sc \textbf{Отчёт\\
			Лабораторная работа №$7$\\
			по дисциплине\\
			"Математическая статистика"}
			\vspace{6mm}
			
		}
		\vspace*{2mm}
		
		
		\begin{flushleft}
			\vspace{4cm}
			\sc Выполнил студент:\\
			\sc Салихов С.Р.\\
			\sc группа: 3630102/70401\\
			\vspace{1cm}
			\sc Проверил:\\
			\sc к.ф-м.н., доцент\\
			\sc Баженов Александр Николавич
			\vspace{20mm}
		\end{flushleft}
	\end{center} 
	\begin{center}
		\vfill {\large\textsc{Санкт-Петербург}}\\ 
		2020 г.
	\end{center}
	
	\newpage
	\pagestyle{plain}
	
	%\begin{center}
	%\begin{abstract} 
	
	%\end{abstract}
	
	%\end{center}
	
	\newpage
	\tableofcontents{}
	\newpage
	\listoffigures
	\newpage
	\listoftables
	\newpage
	
	
	\section{Постановка задачи}
		Сгенерировать выборку объёмом 100 элементов для нормального распределения N(x, 0, 1). По сгенерированной выборке оценить параметры $\mu$ и $\sigma$ нормального закона методом максимального правдоподобия. В качестве основной гипотезы $H_0$ будем считать, что сгенерированное распределение имеет вид N(x, $\hat{\mu}$, $\hat{\sigma}$). Проверить основную гипотезу, используя критерий согласия $\chi^2$. В качестве уровня значимости взять $\alpha$ = 0.05. Привести таблицу вычислений $\chi^2$.
	\section{Теория}
		\subsection{Метод максимального правдоподобия}
		Метод максимального правдоподобия $\--$ метод оценивания неизвестного параметра путём максимимзации функции правдоподобия.
		
		$$\overset{\wedge}{\theta}_{\text{МП}}=argmax \mathbf{L}(x_1,x_2,\ldots,x_n,\theta)
		$$
		
		Где $\mathbf{L}$ это функция правдоподобия, которая представляет собой совместную плотность вероятности независимых случайных величин $X_1,x_2,\ldots,x_n$ и является функцией неизвестного параметра $\theta$
		$$\mathbf{L} = f(x_1,\theta)\cdot f(x_2,\theta)\cdot\cdots\cdot f(x_n,\theta)
		$$
		Оценкой максимального правдоподобия будем называть такое значение $\overset{\wedge}{\theta}_{\text{МП}}$ из множества допустимых значений параметра $\theta,$ для которого функция правдоподобия принимает максимальное значение при заданных $x_1,x_2,\ldots,x_n.$
		
		Тогда при оценивании математического ожидания $m$ и дисперсии $\sigma^2$ нормального распределения $N(m,\sigma)$ получим:
		$$\ln(\mathbf{L})=-\frac{n}{2}\ln(2\pi)-\frac{n}{2}\ln\left(\sigma^2\right)-\frac{1}{2\sigma^2}\sum\limits_{i=1}^n(x_i-m)^2
		$$
		\subsection{Хи-квадрат}
		Разобьём генеральную совокупность на $k$ неперсекающихся подмножеств $\Delta_1, \Delta_2,\ldots, \Delta_k,\;\Delta_i = (a_i,a_{i+1}],$ $p_i = P(X\in\Delta_i),\;i=1,2,\ldots,k\; \--$ вероятность того, что точка попала в $i$ый промежуток.
		
		Так как генеральная совокупность это $\mathbb{R},$ то крайние промежутки будут бесконечными: $\Delta_1=(-\infty,a_1],\;\Delta_k=(a_k,\infty),\;p_i = F(a_i)-F(a_{i-1})$
		
		$n_i\;\--$ частота попадания выборочных элементов в $\Delta_i,\;i=1,2,\ldots,k.$
		
		В случае справедливости гипотезы $H_0$ относительно частоты $\frac{n_i}{n}$ при больших $n$ должны быть близки к $p_i,$ значит в качестве меры имеет смысл взять: 
		
		$$Z = \sum\limits_{i=1}^k\frac{n}{p_i}\left(\frac{n_i}{n}-p_i\right)^2$$
		
		Тогда
		$$\chi^2_B=\sum\limits_{i=1}^k\frac{n}{p_i}\left(\frac{n_i}{n}-p_i\right)^2=\sum\limits_{i=1}^k\frac{(n_i-np_i)^2}{np_i}
		$$
		
		Теорема К.Пирсона. Статистика критерия $\chi^2$ асимптотически распределена по закону $\chi^2$ с k - 1 степенями свободы.
		
	\section{Реализация}
	Для генерации выборки был использован $Python\;3.7$ и модуль numpy. Для отрисовки графиков использовался модуль matplotlib. scipy.stats для обработки функций распределений.
	\newpage
	\section{Результаты}
	Метод максимального правдоподобия:\\
	$$\hat{\mu} \approx 0.03, \hat{\sigma} \approx 0.95$$
	Критерий согласия $\chi^2$:
	\begin{table}[h!]
		
		\caption{Вычисление $\chi^2_B$ при проверке гипотизы $H_0$ о нормальном законе распределения N($x, \hat{\mu} \approx 0.16, \hat{\sigma} \approx 1.04$)}
		\label{tab:my_label}
		\begin{center}
			\vspace{5mm}
			
			\begin{tabular}{|c|c|c|c|c|c|c|}
				\hline
				i & $Delta_i$ , $a_{i - 1}$, $a_i$          &   $n_i$ &   $p_i$ &   $np_i$ &   $n_i-np_i$ &   $\frac{(n_i-np_i)^2}{np_i}$ \\
				\hline
				1 & [-$\infty$, -1.01] & 14 &  0.1562 &    15.62 &        -1.62 &        0.17 \\
				\hline
				2 & [-1.01, -0.37]     & 19 &  0.2004 &    20.04 &        -1.04 &            0.05 \\
				\hline
				3 & [-0.37, 0.28]      &  23 &  0.2517 &    25.17 &        -2.17 &               0.19 \\
				\hline
				4 & [0.28, 0.92]       &   19 &  0.2122 &    21.22 &        -2.22 &           0.23 \\
				\hline
				5 & [0.92, 1.56]       &  14 &  0.1201 &    12.01 &         1.99 &        0.33    \\
				\hline
				6 & [1.56, $\infty$]   & 11 &  0.0594 &     5.94 &         5.06 &         4.32 \\
				\hline
				$\sum$ & -                  &     100 &  1.0000      &   100.00    &         0.00  &                         5.29 = $\chi^2_B$\\
				\hline
			\end{tabular}
		\end{center}
	\end{table}

	Количество промежутков k = 6.\\
	Уровень значимости $\alpha$ = 0.05.\\
	
	\section{Обсуждение}
		По результатам работы, значение критерия согласия Пирсона:\\
		$\chi^2_B = 5.29$. Табличное значение квартиля $\chi^2_{1 - \alpha}(k - 1) = \chi^2_{0.95}(5) = 11.07$.\\
		
		Таким образом, $\chi^2_B < \chi^2_{0.95}(5)$, из этого следует, что основная гипотеза $H_0$(о нормальном законе распределения N($x, \hat{\mu}, \hat{\sigma}$)), на уровне зависимости $\alpha$ = 0.05, соотносится с выборкой.
	\section{Литература}
	
	\href{https://physics.susu.ru/vorontsov/language/numpy.html}{Модуль numpy}\\
	
	\href{https://matplotlib.org/}{Модуль matplotlib}\\
	
	\href{https://www.scipy.org/}{Модуль scipy}\\
	
	\href{https://ru.wikipedia.org/wiki/%D0%9A%D0%B2%D0%B0%D0%BD%D1%82%D0%B8%D0%BB%D0%B8_%D1%80%D0%B0%D1%81%D0%BF%D1%80%D0%B5%D0%B4%D0%B5%D0%BB%D0%B5%D0%BD%D0%B8%D1%8F_%D1%85%D0%B8-%D0%BA%D0%B2%D0%B0%D0%B4%D1%80%D0%B0%D1%82}{Таблица значений $\chi^2$}\\
	
	\section{Приложения}
	
	\href{https://github.com/LuciusGen/Matstat/blob/master/Lab6/Lab7.py}{Код лаборатрной}
	
	\href{https://github.com/LuciusGen/Matstat/blob/master/Lab6/lab7.tex}{Код отчёта}
	
\end{document}