\documentclass[a4]{article}
\pagestyle{myheadings}

%%%%%%%%%%%%%%%%%%%
% Packages/Macros %
%%%%%%%%%%%%%%%%%%%
\usepackage{mathrsfs}


\usepackage{fancyhdr}
\pagestyle{fancy}
\lhead{}
\chead{}
\rhead{}
\lfoot{}
\cfoot{} 
\rfoot{\normalsize\thepage}
\renewcommand{\headrulewidth}{0pt}
\renewcommand{\footrulewidth}{0pt}
\newcommand{\RomanNumeralCaps}[1]
{\MakeUppercase{\romannumeral #1}}

\usepackage{amssymb,latexsym}  % Standard packages
\usepackage[utf8]{inputenc}
\usepackage[russian]{babel}
\usepackage{MnSymbol}
\usepackage{amsmath,amsthm}
\usepackage{indentfirst}
\usepackage{graphicx}%,vmargin}
\usepackage{graphicx}
\graphicspath{{pictures/}} 
\usepackage{verbatim}
\usepackage{color}









\DeclareGraphicsExtensions{.pdf,.png,.jpg}% -- настройка картинок

\usepackage{epigraph} %%% to make inspirational quotes.
\usepackage[all]{xy} %for XyPic'a
\usepackage{color} 
\usepackage{amscd} %для коммутативных диграмм


\newtheorem{Lemma}{Лемма}[section]
\newtheorem{Proposition}{Предложение}[section]
\newtheorem{Theorem}{Теорема}[section]
\newtheorem{Corollary}{Следствие}[section]
\newtheorem{Remark}{Замечание}[section]
\newtheorem{Definition}{Определение}[section]
\newtheorem{Designations}{Обозначение}[section]




%%%%%%%%%%%%%%%%%%%%%%%% 
%Сношение с оглавлением% 
%%%%%%%%%%%%%%%%%%%%%%%% 
\usepackage{tocloft} 
\renewcommand{\cftdotsep}{2} %частота точек
\renewcommand\cftsecleader{\cftdotfill{\cftdotsep}}
\renewcommand{\cfttoctitlefont}{\hspace{0.38\textwidth} \LARGE\bfseries} 
\renewcommand{\cftsecaftersnum}{.}
\renewcommand{\cftsubsecaftersnum}{.}
\renewcommand{\cftbeforetoctitleskip}{-1em} 
\renewcommand{\cftaftertoctitle}{\mbox{}\hfill \\ \mbox{}\hfill{\footnotesize Стр.}\vspace{-0.5em}} 
\renewcommand{\cftsubsecfont}{\hspace{1pt}} 
\renewcommand{\cftparskip}{3mm} %определяет величину отступа в оглавлении
\setcounter{tocdepth}{5} 




\addtolength{\textwidth}{0.7in}
\textheight=630pt
\addtolength{\evensidemargin}{-0.4in}
\addtolength{\oddsidemargin}{-0.4in}
\addtolength{\topmargin}{-0.4in}

\newcommand{\empline}{\mbox{}\newline} 
\newcommand{\likechapterheading}[1]{ 
	\begin{center} 
		\textbf{\MakeUppercase{#1}} 
	\end{center} 
	\empline} 

\makeatletter 
\renewcommand{\@dotsep}{2} 
\newcommand{\l@likechapter}[2]{{\bfseries\@dottedtocline{0}{0pt}{0pt}{#1}{#2}}} 
\makeatother 
\newcommand{\likechapter}[1]{ 
	\likechapterheading{#1} 
	\addcontentsline{toc}{likechapter}{\MakeUppercase{#1}}} 





\usepackage{xcolor}
\usepackage{hyperref}
\definecolor{linkcolor}{HTML}{000000} % цвет ссылок
\definecolor{urlcolor}{HTML}{AA1622} % цвет гиперссылок

\hypersetup{pdfstartview=FitH,  linkcolor=linkcolor,urlcolor=urlcolor, colorlinks=true}



\def \newstr {\medskip \par \noindent} 



\begin{document}
	\def\contentsname{\LARGE{Содержание}}
	\thispagestyle{empty}
	\begin{center} 
		\vspace{2cm} 
		{\Large \sc Санкт-Петербургский Политехнический Университет}\\
		\vspace{2mm}
		{\Large\sc Петра Великого}\\
		\vspace{1cm}
		{\large \sc Институт прикладной математики и механики\\ 
			\vspace{0.5mm}
			\textsc{}}\\ 
		\vspace{0.5mm}
		{\large\sc Кафедра $"$Прикладная математика$"$}\\
		\vspace{15mm}
		
		
		{\sc \textbf{Отчёт\\
			Лабораторная работа №$8$\\
			по дисциплине\\
			"Математическая статистика"}
			\vspace{6mm}
			
		}
		\vspace*{2mm}
		
		
		\begin{flushleft}
			\vspace{4cm}
			\sc Выполнил студент:\\
			\sc Салихов С.Р.\\
			\sc группа: 3630102/70401\\
			\vspace{1cm}
			\sc Проверил:\\
			\sc к.ф-м.н., доцент\\
			\sc Баженов Александр Николавич
			\vspace{20mm}
		\end{flushleft}
	\end{center} 
	\begin{center}
		\vfill {\large\textsc{Санкт-Петербург}}\\ 
		2020 г.
	\end{center}
	
	\newpage
	\pagestyle{plain}
	
	%\begin{center}
	%\begin{abstract} 
	
	%\end{abstract}
	
	%\end{center}
	
	\newpage
	\tableofcontents{}
	\newpage
	\listoffigures
	\newpage
	\listoftables
	\newpage
	
	
	\section{Постановка задачи}
		Для двух выборок размерами 20 и 100 элементов, сгенерированных согласно нормальному закону N(x, 0, 1), для параметров положения и масштаба построить асимптотически нормальные интервальные оценки на основе точечных оценок метода максимального правдоподобия и классические интервальные оценки на основе статистик $\chi^2$ и Стьюдента. В качестве параметра надёжности взять $\gamma$ = 0.95.
	\section{Теория}
		\subsection{Доверительные интервалы для параметров нормального распредееления}
		Оценкой максимального правдоподобия для математического ожидания  является среднее арифметическое: $\mu=\frac{1}{n}\sum\limits_{i=1}^nx_i.$
		
		Оценка максимального правдоподобия для дисперсии вычисляется по формуле: $\sigma^2 = \frac{1}{n}\sum\limits_{i=1}^n(x_i-\overline{x})^2.$
		
		Доверительным интервалом или интервальной оценкой числовой характеристики или параметра распределения $\theta$ с доверительной вероятностью $\gamma$ называется интервал со случайными границами $(\theta_1,\theta_2),$ содержащий параметр $\theta$ с вероятностью $\gamma$.
		
		Функция распределения Стьюдента:
		$$
		T = \sqrt{n-1}\frac{\overline{x}-\mu}{\delta}
		$$
		
		Функция плотности распределения $\chi^2$:
		$$
		f(x) = \begin{cases}
		0,&x\leq 0\\
		\frac{1}{2^\frac{n}{2}\Gamma\left(\frac{n}{2}\right)}x^{\frac{n}{2}-1}e^{-\frac{x}{2}},& x>0
		\end{cases}
		$$
		Интервальная оценка математического ожидания:
		$$
		P=\left(\overline{x}-\frac{\sigma t_{1-\frac{a}{2}}(n-1)}{\sqrt{n-1}}<\mu<\overline{x}+\frac{\sigma t_{1-\frac{a}{2}}(n-1)}{\sqrt{n-1}}\right) = \gamma,
		$$
		где $t_{1-\frac{a}{2}}\;\--$ квантиль распределения Стьюдента порядка $1-\frac{a}{2}.$
		
		Интервальная оценка дисперсии:
		$$
		P=\left(\frac{\sigma\sqrt{n}}{\sqrt{\chi^2_{1-\frac{a}{2}}(n-1)}}<\sigma<\frac{\sigma\sqrt{n}}{\sqrt{\chi^2_\frac{a}{2}(n-1)}}\right) = \gamma,
		$$
		где $\chi_{1-\frac{a}{2}}^2,\;\chi_\frac{a}{2}^2\;\--$ квантили распределения Стьюдента порядков $1-\frac{a}{2}$ и $\frac{a}{2}$ соответственно.
		\subsection{Доверительные интервалы для математического ожидания m и среднего квадратического отклонения $\sigma$ произвольного распределения при большом объёме выборки. Асимптотический подход}
		
		Асимптотическая интервальная оценка математического ожидания:
		$$P = \left(\overline{x}-\frac{\sigma u_{1-\frac{a}{2}}}{\sqrt{n}}<m<\overline{x}+\frac{\sigma u_{1-\frac{a}{2}}}{\sqrt{n}}\right)=\gamma,
		$$
		где $u_{1-\frac{a}{2}}\;\--$ квантиль нормального распределения $N(x,0,1)$ порядка $1-\frac{a}{2}.$
		
		$$\sigma(1 - 0.5u_{1 - \alpha/2} \sqrt{e + 2}/ \sqrt{n}) < \sigma < \sigma(1 + 0.5u_{1 - \alpha/2} \sqrt{e + 2}/ \sqrt{n})$$
		
	\section{Реализация}
	Для генерации выборки был использован $Python\;3.7$ и модуль numpy. Для отрисовки графиков использовался модуль matplotlib. scipy.stats для обработки функций распределений.
	
	\section{Результаты}
		\begin{table}[h!]
			
			\caption{Доверительные интервалы для параметров нормального распределения}
			\label{tab:my_label}
			\begin{center}
				\vspace{5mm}
				
				\begin{tabular}{|c|c|c|}
					\hline
					n = 20 & m & $\sigma$\\
					\hline
					& -0.48 < m < 0.54 & 0.83 < $\sigma$ < 1.59\\ 
					\hline
					& &\\
					\hline
					n = 100 & m & $\sigma$\\
					\hline
					& -0.25 < m < 0.15 & 0.9 < $\sigma$ < 1.19\\
					\hline
				\end{tabular}
			\end{center}
		\end{table}
		
		\begin{table}[h!]
			
			\caption{Доверительные интервалы для параметров произвольного распределения. Асимптотический подход}
			\label{tab:my_label}
			\begin{center}
				\vspace{5mm}
				
				\begin{tabular}{|c|c|c|}
					\hline
					n = 20 & m & $\sigma$\\
					\hline
					& -0.43 < m < 0.49 & 0.93 < $\sigma$ < 1.25\\ 
					\hline
					& & \\
					\hline
					n = 100 & m & $\sigma$\\
					\hline
					& -0.25 < m < 0.15 & 0.95 < $\sigma$ < 1.09\\
					\hline
				\end{tabular}
			\end{center}
		\end{table}

	\section{Обсуждение}
		Генеральные характеристики m = 0, $\sigma$ = 0 накрываются построенными доверительными интервалами.\\
		
		По полученным результатам видно, что лучший результат достигается на выборках большого объема.\\
		При сравнении результатов при объеме n = 20, видим, что интервал меньше для парамаетров произвольного распределения.
		
	\section{Литература}
	
	\href{https://physics.susu.ru/vorontsov/language/numpy.html}{Модуль numpy}\\
	
	\href{https://www.scipy.org/}{Модуль scipy}\\
	
	\href{https://ru.wikipedia.org/wiki/%D0%9D%D0%BE%D1%80%D0%BC%D0%B0%D0%BB%D1%8C%D0%BD%D0%BE%D0%B5_%D1%80%D0%B0%D1%81%D0%BF%D1%80%D0%B5%D0%B4%D0%B5%D0%BB%D0%B5%D0%BD%D0%B8%D0%B5}{Нормальное распределение}\\
	
	\href{http://mit.spbau.ru/sewiki/images/a/a1/Cis.pdf}{Доверительные интервалы}\\
	
	
	\section{Приложения}
	
	\href{https://github.com/LuciusGen/Matstat/blob/master/Lab8/Lab8.py}{Код лаборатрной}
	
	\href{https://github.com/LuciusGen/Matstat/blob/master/Lab8/lab8.tex}{Код отчёта}
	
\end{document}